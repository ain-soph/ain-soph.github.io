% LaTeX file for resume 
% This file uses the resume document class (res.cls)
\documentclass{resume} 

\usepackage[T1]{fontenc}
\usepackage{crimson}
\usepackage{enumitem}

\usepackage{url}

\newsectionwidth{0pt}  % So the text is not indented under section headings
\usepackage{fancyhdr}  % use this package to get a 2 line header
\renewcommand{\headrulewidth}{0pt} % suppress line drawn by default by fancyhdr
\setlength{\headheight}{24pt} % allow room for 2-line header
\setlength{\headsep}{24pt}  % space between header and text
\setlength{\headheight}{24pt} % allow room for 2-line header
\pagestyle{fancy}     % set pagestyle for document


\rhead{{\it R. Pang}} % put text in header (right side)
\rfoot{{\it p. \thepage}}
\cfoot{}                                  % the foot is empty
\topmargin=-0.5in % start text higher on the page


\begin{document}
\thispagestyle{empty} % this page has no header  
\name{REN PANG\\[11pt]}% the \\[12pt] adds a blank line after name


\address{\centerline{College of Information Sciences and Technology, the Pennsylvania State University}\\
\centerline{{\it Email}: rbp5354@psu.edu \quad {\it Tel}: (484) 747-2401 \quad {\it Web}: \url{https://ain-soph.github.io}}}      
                                      
% \address{{\bf Permanent Address} \\ 72 Trolley Lane \\ Utica, NY 13502\\
%   (315) 493-8621 }


\begin{resume}
 
\section{A. Research Interests}
\vspace{8pt} % provide vertical space between section title and contents
Deep Learning Security: Adversarial Robustness; Neural Backdoors

AutoML: Neural Architecture Search; Auto Augment

Others: Lifelong Learning; Dataset Condensation
% My current research focuses on developing rigorous yet practical solutions to improve the security assurance, privacy preservation, and decision-making transparency of AI systems and applications.    % TODO

\vspace{0.2in} 
\section{B. Education Background} 
\vspace{8pt}

Ph.D., Information Sciences and Technology, Pennsylvania State University  \hfill 2019--present

Ph.D., Computer Science and Engineering, Lehigh University (transferred)  \hfill 2018--2019

BSc., Mathematics, Nankai University   \hfill 2014-2018

\vspace{0.2in} 
\section{C. Publications}
\vspace{24pt}

\begin{enumerate}[labelsep=15pt, parsep=8pt, resume]

\item TrojanZoo: Towards Unified, Holistic, and Practical Evaluation of Neural Backdoors,\\
\textbf{R. Pang}, Z. Zhang, X. Gao, Z. Xi, S. Ji, P. Cheng, and T. Wang,\\
Proceedings of {\it the IEEE European Symposium on Security and Privacy} (EuroS\&P), 2022.

\item On the Security Risks of AutoML,\\
\textbf{R. Pang}, Z. Xi, S. Ji, X. Luo, and T. Wang,\\
Proceedings of {\it the USENIX Security Symposium} (SECURITY), 2022.

\item Graph Backdoor,\\
Z. Xi, \textbf{R. Pang}, S. Ji, and T. Wang,\\
Proceedings of {\it the USENIX Security Symposium} (SECURITY), 2021.

\item i-Algebra: Towards Interactive Interpretability of Deep Neural Networks,\\
X. Zhang, \textbf{R. Pang}, S. Ji, F. Ma, and T. Wang,\\
Proceedings of {\it the AAAI Conference on Artificial Intelligence} (AAAI), 2021.

\item AdvMind: Inferring Adversary Intent of Black-Box Attacks,\\
\textbf{R. Pang}, X. Zhang, S. Ji, X. Luo, and T. Wang,\\
Proceedings of {\it the ACM SIGKDD Conference on Knowledge Discovery and Data Mining} (KDD), 2020.

\item A Tale of Evil Twins: Adversarial Inputs versus Poisoned Models,\\
\textbf{R. Pang}, H. Shen, X. Zhang, S. Ji, Y. Vorobeychik, X. Luo, A. Liu, and T. Wang,\\
Proceedings of {\it the ACM Conference on Computer and Communications Security} (CCS), 2020.

\end{enumerate}

\newpage

\section{D. Open-Sourced Projects}
\vspace{8pt}

\section{Owner}
\vspace{8pt}

AlpsPlot \\
\url{https://github.com/ain-soph/alpsplot}

TrojanZoo: Towards Unified, Holistic, and Practical Evaluation of Neural Backdoors \\
\url{https://github.com/ain-soph/trojanzoo}

TrojanZoo Sphinx Theme \\
\url{https://github.com/ain-soph/trojanzoo_sphinx_theme}

\section{Contributor}
\vspace{8pt}

matplotlib; pytorch\_sphinx\_theme; sphinxcontrib-katex; torchvision


\vspace{0.2in} 
\section{E. Teaching Assistant Experiences} 
\vspace{8pt}

CYBER 497: Machine Learning Security (Penn State), 2020 Spring

CSE 017: Structured Programming and Data Structures (Lehigh), 2018 Fall


\vspace{0.2in} 
\section{F. Technical Skills} 
\vspace{8pt}

\section{Language}
\vspace{8pt}

Python; Java; C++; MatLab; Bash; LaTeX; HTML; JavaScript

\section{Package}
\vspace{8pt}

pytorch; matplotlib; sphinx; jinja; pytest

\section{Tools}
\vspace{8pt}

Auto CI; Docker; GitHub Actions

\end{resume} 
\end{document}













