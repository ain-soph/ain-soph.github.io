% -------------------------
% Resume in Latex
% Author : Ren Pang
% Based off of: https://github.com/jakeryang/resume
% License : MIT
% ------------------------

\documentclass[letterpaper,11pt]{article}

\usepackage{latexsym}
\usepackage[empty]{fullpage}
\usepackage{titlesec}
\usepackage[usenames,dvipsnames]{color}
\usepackage{enumitem}
\usepackage{fancyhdr}
\usepackage{url}

% fontawesome
\usepackage{fontawesome5}

% fixed width
\usepackage[scale=0.90,lf]{FiraMono}

% light-grey
\definecolor{light-grey}{gray}{0.83}
\definecolor{dark-grey}{gray}{0.3}
\definecolor{text-grey}{gray}{.08}

\DeclareRobustCommand{\ebseries}{\fontseries{eb}\selectfont}
\DeclareTextFontCommand{\texteb}{\ebseries}

% custom underilne
\usepackage{contour}
\usepackage[normalem]{ulem}
\renewcommand{\ULdepth}{1.8pt}
\contourlength{0.8pt}
\newcommand{\myuline}[1]{%
  \uline{\phantom{#1}}%
  \llap{\contour{white}{#1}}%
}


% custom font: helvetica-style
\usepackage{tgheros}
\renewcommand*\familydefault{\sfdefault} 
%% Only if the base font of the document is to be sans serif
\usepackage[T1]{fontenc}


\pagestyle{fancy}
\fancyhf{} % clear all header and footer fields
\fancyfoot{}
\renewcommand{\headrulewidth}{0pt}
\renewcommand{\footrulewidth}{0pt}

% Adjust margins
\addtolength{\oddsidemargin}{-0.5in}
\addtolength{\evensidemargin}{0in}
\addtolength{\textwidth}{1in}
\addtolength{\topmargin}{-.5in}
\addtolength{\textheight}{1.0in}


\titleformat{\section}{
    \bfseries \vspace{2pt} \raggedright \large % header section
}{}{0em}{}[\color{light-grey} {\titlerule[2pt]} \vspace{-4pt}]


\begin{document}

\begin{center}
    \textbf{\Huge Ren Pang} \\ \vspace{5pt}
    \small \faPhone* \texttt{(484)747--2401} \hspace{1pt} $|$
    \hspace{1pt} \faEnvelope\hspace{2pt}\texttt{ain-soph@live.com} \hspace{1pt} $|$ 
    \hspace{1pt} \faGithub\hspace{2pt}\texttt{ain-soph.github.io}
    \\ \vspace{-10pt}
\end{center}

\section{RESEARCH INTEREST}
My research focuses on understanding and tackling the challenges arising in the advances of deep learning and artificial intelligence in general, especially about security risks in image classification task.


\section{EDUCATION}
\begin{tabular*}{\linewidth}{llr@{\extracolsep{\fill}}r}
    Ph.D., & Information Sciences and Technology, & Pennsylvania State University & 2019--2023 \\
    BSc., & Mathematics, & Nankai University & 2014--2018
\end{tabular*}
% Ph.D., Computer Science and Engineering, Lehigh University (transferred)  \hfill 2018--2019 \\


\section{EXPERIENCE}

Software Engineer (Intern), Meta  \hfill 2022 Summer \\
The work focuses on in-stream ad breaks demonetization during creator onboarding, which detects malicious creators/pages that violate Facebook regulations. During the development, I migrate fix several existing bugs which introduce good but incorrect metric results, extend labels from binary violation to concrete violated policies. SQL and FBlearner are used in the work.

\section{PUBLICATION}

\begin{enumerate}[labelsep=15pt, parsep=-4pt]

    \item A Tale of Evil Twins:\@ Adversarial Inputs versus Poisoned Models,\\
    \textbf{R. Pang}, H. Shen, X. Zhang, S. Ji, Y. Vorobeychik, X. Luo, A. Liu, and T. Wang,\\
    Proceedings of {\it the ACM Conference on Computer and Communications Security\/} (\textbf{CCS}), 2020.
    
    \item AdvMind:\@ Inferring Adversary Intent of Black-Box Attacks,\\
    \textbf{R. Pang}, X. Zhang, S. Ji, X. Luo, and T. Wang,\\
    Proceedings of {\it the ACM SIGKDD Conference on Knowledge Discovery and Data Mining\/} (\textbf{KDD}), 2020.
    
    \item i-Algebra:\@ Towards Interactive Interpretability of Deep Neural Networks,\\
    X. Zhang, \textbf{R. Pang}, S. Ji, F. Ma, and T. Wang,\\
    Proceedings of {\it the AAAI Conference on Artificial Intelligence\/} (\textbf{AAAI}), 2021.
    
    \item Graph Backdoor,\\
    Z. Xi, \textbf{R. Pang}, S. Ji, and T. Wang,\\
    Proceedings of {\it the USENIX Security Symposium\/} (\textbf{USENIX}), 2021.
    
    \item On the Security Risks of AutoML,\\
    \textbf{R. Pang}, Z. Xi, S. Ji, X. Luo, and T. Wang,\\
    Proceedings of {\it the USENIX Security Symposium\/} (\textbf{USENIX}), 2022.
    
    \item TrojanZoo:\@ Towards Unified, Holistic, and Practical Evaluation of Neural Backdoors,\\
    \textbf{R. Pang}, Z. Zhang, X. Gao, Z. Xi, S. Ji, P. Cheng, and T. Wang,\\
    Proceedings of {\it the IEEE European Symposium on Security and Privacy\/} (\textbf{EuroS\&P}), 2022.
    
    \item The Dark Side of AutoML:\@ Towards Architectural Backdoor Search,\\
    \textbf{R. Pang}, C. Li, Z. Xi, S. Ji, T. Wang,\\
    Arxiv Preprint, 2022.
    
    \item Demystifying Self-supervised Trojan Attacks,\\
    C. Li, \textbf{R. Pang}, Z. Xi, T. Du, S. Ji, Y. Yao, T. Wang,\\
    Arxiv Preprint, 2022.
    
    \item Reasoning over Multi-view Knowledge Graphs,\\
    Z. Xi, \textbf{R. Pang}, C. Li, T. Du, S. Ji, F. Ma, T. Wang,\\
    Arxiv Preprint, 2022.
    
    \end{enumerate}
    

\section{OPEN-SOURCED ARTIFACT}

\begin{enumerate}[labelsep=15pt, parsep=4pt]

    \item TrojanZoo\\
    \url{https://ain-soph.github.io/trojanzoo} \\
    A universal, flexible PyTorch platform to conduct security analysis of attacks and defenses (e.g., adversarial evasion, backdoor injection, model poisoning) on deep neural network models. 
    
    \item TrojanZoo Sphinx Theme \\
    \url{https://ain-soph.github.io/trojanzoo_sphinx_theme} \\
    A light-weight, customizable theme that generalizes pytorch\_sphinx\_theme.
    
    \item AlpsPlot \\
    \url{https://ain-soph.github.io/alpsplot} \\
    A customizable Python plotting library based on matplotlib.

\end{enumerate}

\end{document}
