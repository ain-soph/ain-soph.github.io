% -------------------------
% Resume in Latex
% Author : Ren Pang
% Based off of: https://github.com/jakeryang/resume
% License : MIT
% ------------------------

\documentclass[letterpaper,11pt]{article}

\usepackage{latexsym}
\usepackage[empty]{fullpage}
\usepackage{titlesec}
\usepackage[usenames,dvipsnames]{color}
\usepackage{enumitem}
\usepackage{fancyhdr}
\usepackage{hyperref}

% fontawesome
\usepackage{fontawesome5}

% fixed width
\usepackage[scale=0.90,lf]{FiraMono}

% light-grey
\definecolor{light-grey}{gray}{0.83}
\definecolor{dark-grey}{gray}{0.3}
\definecolor{text-grey}{gray}{.08}

\DeclareRobustCommand{\ebseries}{\fontseries{eb}\selectfont}
\DeclareTextFontCommand{\texteb}{\ebseries}

% custom underline
\usepackage{contour}
\usepackage[normalem]{ulem}
\renewcommand{\ULdepth}{1.8pt}
\contourlength{0.8pt}
\newcommand{\myuline}[1]{%
  \uline{\phantom{#1}}%
  \llap{\contour{white}{#1}}%
}

% Render a link with its symbol
% Usage; \sociallink{<icon>}{<label>}
\newcommand{\sociallink}[3]{\mbox{{#1}\hspace{0.5em}\href{#2}{#3}\hspace{1em}}}
\newcommand*{\website}[1]{\sociallink{\faGithub}{#1}{#1}}

% custom font: helvetica-style
\usepackage{tgheros}
% \renewcommand*\familydefault{\sfdefault} 
%% Only if the base font of the document is to be sans serif
\usepackage[T1]{fontenc}


\pagestyle{fancy}
\fancyhf{} % clear all header and footer fields
\fancyfoot{}
\renewcommand{\headrulewidth}{0pt}
\renewcommand{\footrulewidth}{0pt}

% Adjust margins
\addtolength{\oddsidemargin}{-0.5in}
\addtolength{\evensidemargin}{0in}
\addtolength{\textwidth}{1in}
\addtolength{\topmargin}{-.5in}
\addtolength{\textheight}{1.0in}

\titleformat{\section}{
    \bfseries \vspace{2pt} \raggedright \large % header section
}{}{0em}{}[\color{dark-grey} {\titlerule[1pt]} \vspace{-4pt}]

\begin{document}

\begin{center}
    \textbf{\Huge Ren Pang} \\ \vspace{5pt}
    \small \faPhone* (484)747--2401 \hspace{1pt} $|$
    \hspace{1pt} \faEnvelope\hspace{2pt}ain-soph@live.com \hspace{1pt} $|$ 
    \hspace{1pt} \faGithub\hspace{2pt}ain-soph.github.io
    \\ \vspace{0pt}
\end{center}

\noindent
My research focuses on developing safe, robust and resilient machine/deep learning applications. Experienced in addressing the security concerns in image classification, AutoML, etc.

\section{EDUCATION}
Ph.D. \hspace{1em} \textit{Information Sciences and Technology} \hspace{1em} Pennsylvania State University \hfill 2019--2023 \\
B.Sc. \hspace{1em} \textit{Mathematics} \hspace{11.8em} Nankai University \hfill 2014--2018
% Ph.D., Computer Science and Engineering, Lehigh University (transferred)  \hfill 2018--2019 \\

\section{WORK EXPERIENCE}
Software Engineer (Intern), \textit{Meta}  \hfill 2022 Summer\vspace{3pt}\\
\textbf{Pages and Groups Integrity}\\
Introduce new classification model for malicious page detection. It mitigates the impact of incorrect label annotation, and provides interpretable classification outputs for better user experience. \vspace{0.8em}\\
\textbf{TorchVision}\\
Provide the official TorchVision implementation of SwinTransformerV2.

\section{PUBLICATIONS}

\begin{enumerate}[labelsep=15pt, parsep=-4pt]

    \item A Tale of Evil Twins:\@ Adversarial Inputs versus Poisoned Models,\\
    \textbf{R. Pang}, H. Shen, X. Zhang, S. Ji, Y. Vorobeychik, X. Luo, A. Liu, and T. Wang,\\
    Proceedings of {\it the ACM Conference on Computer and Communications Security\/} (\textbf{CCS}), 2020.
    
    \item AdvMind:\@ Inferring Adversary Intent of Black-Box Attacks,\\
    \textbf{R. Pang}, X. Zhang, S. Ji, X. Luo, and T. Wang,\\
    Proceedings of {\it the ACM SIGKDD Conference on Knowledge Discovery and Data Mining\/} (\textbf{KDD}), 2020.
    
    \item i-Algebra:\@ Towards Interactive Interpretability of Deep Neural Networks,\\
    X. Zhang, \textbf{R. Pang}, S. Ji, F. Ma, and T. Wang,\\
    Proceedings of {\it the AAAI Conference on Artificial Intelligence\/} (\textbf{AAAI}), 2021.
    
    \item Graph Backdoor,\\
    Z. Xi, \textbf{R. Pang}, S. Ji, and T. Wang,\\
    Proceedings of {\it the USENIX Security Symposium\/} (\textbf{USENIX}), 2021.
    
    \item On the Security Risks of AutoML,\\
    \textbf{R. Pang}, Z. Xi, S. Ji, X. Luo, and T. Wang,\\
    Proceedings of {\it the USENIX Security Symposium\/} (\textbf{USENIX}), 2022.
    
    \item TrojanZoo:\@ Towards Unified, Holistic, and Practical Evaluation of Neural Backdoors,\\
    \textbf{R. Pang}, Z. Zhang, X. Gao, Z. Xi, S. Ji, P. Cheng, and T. Wang,\\
    Proceedings of {\it the IEEE European Symposium on Security and Privacy\/} (\textbf{EuroS\&P}), 2022.
    
    \item The Dark Side of AutoML:\@ Towards Architectural Backdoor Search,\\
    \textbf{R. Pang}, C. Li, Z. Xi, S. Ji, T. Wang,\\
    Proceedings of {\it the International Conference on Learning Representations\/} (\textbf{ICLR}), 2023.
    
    \item Demystifying Self-supervised Trojan Attacks,\\
    C. Li, \textbf{R. Pang}, Z. Xi, T. Du, S. Ji, Y. Yao, T. Wang,\\
    Arxiv Preprint, 2022.
    
    \item Reasoning over Multi-view Knowledge Graphs,\\
    Z. Xi, \textbf{R. Pang}, C. Li, T. Du, S. Ji, F. Ma, T. Wang,\\
    Arxiv Preprint, 2022.

    \item On the Difficulty of Defending Contrastive Learning against Backdoor Attacks,\\
    C. Li, \textbf{R. Pang}, B. Cao, Z. Xi, J. Chen, S. Ji, T. Wang,\\
    Arxiv Preprint, 2023.

    \end{enumerate}
    

\section{OPEN-SOURCE CONTRIBUTION}

\begin{enumerate}[labelsep=15pt, parsep=4pt]

    \item \textbf{TrojanZoo} \hfill \website{https://github.com/ain-soph/trojanzoo}
    \vspace{0.3em}\\
    Offer a universal, flexible PyTorch platform to conduct security analysis of attacks and defenses on deep neural network models.
    
    \item \textbf{TorchVision.SwinTransformerV2} \hfill
    \website{https://github.com/pytorch/vision/pull/6246}
    \vspace{0.3em}\\
    Provide TorchVision official implementation of SwinTransformerV2.

    \item \textbf{TorchVision.AutoAugmentation} \hfill
    \website{https://github.com/pytorch/vision/pull/6609}
    \vspace{0.3em}\\
    Provide TorchVision official implementation of AutoAugmentation for object detection. This work is based on the next generation PyTorch APIs.

    \item \textbf{Matplotlib.Text} \hfill
    \website{https://github.com/matplotlib/matplotlib/pull/20101}
    \vspace{0.3em}\\
    Fix Text class bug when font argument is provided without math\_fontfamily.
    
    
\end{enumerate}

\section{TEACHING EXPERIENCE}
CSE 017: Structured Programming and Data Structures, \textit{Lehigh University}  \hfill 2018 Fall \vspace{3pt}\\

\section{SELECTED PROJECTS}
\textbf{Mutual Reinforcement of Adversarial Inputs and Poisoned Models}\\
The project presents a new unified attack model called ``IMC'' that jointly optimizes adversarial inputs and poisoned models. It shows that there are mutual reinforcement effects between the two attack vectors and enables a large design spectrum for the adversary to enhance existing attacks such as backdoor attacks. It also discusses potential countermeasures and technical challenges, pointing to promising research directions. \vspace{0.8em}\\
\textbf{Inferring Malicious Intent of Adversarial Machine Learning}\\
The project presents a new class of estimation models that infer the intent of black-box adversarial attacks in a robust and prompt manner by taking into account fake queries and proactively soliciting subsequent queries to maximize exposure of the adversary's intent. \vspace{0.8em}\\
\textbf{Exploring Vulnerabilities of AutoML Architectures}\\
This project examines the potential security risks of using neural architecture search (NAS) in machine learning systems. The study finds that NAS-generated models are more vulnerable to malicious manipulations compared to manually designed models. The study also provides explanations for this vulnerability, such as early convergence during training, and suggests potential remedies such as increasing cell depth or suppressing skip connections. \vspace{0.8em}\\


\end{document}
